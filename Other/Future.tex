Weak logical link between Western entities giving fewer concessions and the conclusion that recipients should diversify borrowing

We're implicitly assuming that loans are sufficiently easy for African countries to obtain from both China and West. Big problem
%%%%%%%%%%%%%%%%%%
need to hit hard the "suggestion" from the literature. what exactly is suggested and how do you synthesize that into expectations that seem unobserved thus far?

Really hit the time inconsistency argument. what should countries gain in the long run? why do temporally close considerations disrupt their ability to obtain this?

How does your explanation interact with alternatives? Aren't there some major differences between Cabo Verde and Djibouti, for example? What else might explain why China is a large lender in the later but not the former, relatively speaking?