\section*{Introduction}
The People's Republic of China (China) established the Belt and Road Initiative (BRI) in 2013, a project aimed at increasing both China's global influence and domestic economic growth. Since then, China has become the ``lender of first resort'' for developing countries \parencite[1]{dreher2022}.

There has been an abundance of scholarly work attempting to understand the potential effects of China's recent prominence in the international financing and aid arena. Most, if not all, of this literature focuses on how the behavior of other aid donors has been affected (e.g., \cite{humphrey2019}; \cite{kilama2016a}) or how the effectiveness of their aid has changed (e.g., \cite{blair2022}; \cite{gehring2022}). Essentially, this work has explored how the supply side of foreign aid has been influenced by China's aid involvement. 

However, there has been little research on how Chinese aid affects the demand side of foreign aid (i.e., recipient behavior), despite the supply side literature implying that the presence of Chinese aid is indeed consequential for recipients. In particular, it appears that the democratizing effects of aid from the Organization for Economic Cooperation and Development (OECD) has been reduced in light of Chinese aid (\cite{li2017a}) and that the World Bank provides fewer conditions for it's loans to potentially compete with China (\cite{hernandez2017}).

Given this seemingly enhanced foreign aid competition, I argue that recipient behavior is no longer a trivial question, as the conventional wisdom assumes. It should be plausible that recipients engage in some strategy to maximize the financial package they receive from donors, possibly diversifying their aid diet to include Chinese funds and gain leverage on Western donors. 

To explore the possible strategic behavior of international finance recipients in this situation, I build a formal bargaining model below. The model attempts to capture the flexibility of Chinese lending practices and the results of Chinese competition with more traditional, OECD donors.

Empirically, I focus on evaluating the model's implications in Africa using available data on Chinese and Western loans. Data availability for Chinese aid limits the period of interest from 2000-2017.

\section*{Theory}
\subsection*{Model}
In an attempt to model the effects of having more than one donor on recipient behavior, especially when donors display heterogeneity in their loaning practices, I formalize a bargaining interaction between a recipient country and two types of donors. Figure 1 illustrates the extensive-form of the interaction and can be found in the appendix.  

Bargaining occurs over two periods, with the first donor ($D_1$) making an initial loan offer ($x_1$) to the recipient ($R$). If the recipient accepts the loan, the first donor gets a payoff ($1-x_1$) that corresponds to the expected benefits from the conditions of the loan (e.g., political reform, trade liberalization). The recipient gains the value of the loan, $x_1$.

If the recipient rejects the offer, then the second donor ($D_2$) has an opportunity to proffer a loan ($x_2$). Acceptance by the recipient in this second period yields similar payoffs to those in the first (i.e., $1-x_2$ and $x_2$), except now there are two additional parameters, $\delta$ and $\gamma$, such that the recipient's value of the loan in the second period totals to $\delta \gamma x_2$. Consequently, the donor would receive $1-\delta \gamma x_2$ if the offer is accepted.

The parameter $\delta$ commonly appears in multi-period bargaining models and is referred to as a discount factor. The substantive interpretation in this scenario is that recipients shopping for loans have some time sensitive reason for wanting a loan, such as an economic crisis. The smaller that $\delta$ is, the more harmful it is for recipients to refuse the first loan offer they receive and wait for another. Parameter $\gamma$ relates to the level of uncertainty surrounding the conditionality of the loan in period two, where smaller values of $\gamma$ denote greater ambiguity. This allows the model to capture variation in the ``officialness'' of aid, or the heterogeneity between donors who are more restricted and transparent in the sort of conditions they can impose versus those who have more freedom to set conditions and the clarity surrounding them.

The outcome of the game is that the first donor offers a marginally better loan to the recipient than the second donor $ceteris\;paribus$ (see appendix). In other words, in almost all cases the presence of a second credible donor induces competition that benefits the recipient, regardless of the exact nature of donor two's lending practices. Unless the recipient is in perfectly extreme need (i.e., $\delta=0$) or the second donor is perfectly shady about it's loans (i.e., $\gamma=0$), then a diversified set of donors induces loans that are more favorable to the recipient.

\subsection*{Implications}
The first donor's offer is largest, though, when the severity of recipient need is low (high $\delta$) and the second donor is anticipated to implement clear conditions (high $\gamma$). This implies the comparative statics that as severity of need increases (i.e., $\delta$ gets smaller) or clarity of conditionality decreases (i.e., $\gamma$ gets smaller), loan value decreases for the recipient. This analysis produces the following hypotheses, conditional on the presence of a diverse set of donors:
\begin{align*}
    &\textbf{Recipient Need is Inversely Related to Loan Value}\\
    H_{1_a}&:\;\text{An increase in severity of recipient need will decrease the amount of money offered via loans}\\
    H_{1_b}&:\;\text{A decrease in severity of recipient need will increase the amount of money offered via loans}\\
    &\textbf{Condition Clarity is Directly Related to Loan Value}\\
    H_{2_a}&:\;\text{An increase in the clarity of loan conditionality will increase the amount of money offered via loans}\\
    H_{2_b}&:\;\text{A decrease in the clarity of loan conditionality will decrease the amount of money offered via loans}\\
\end{align*}
In other words, I expect to find the following relationship:

\subsection*{Table 1}
\begin{tabular}{c|c c }
    & Severity of Recipient Need & Clarity of Loan Conditionality\\
    \hline
    High & Low Loan Value & High Loan Value\\
    Low & High Loan Value & Low Loan Value\\
\end{tabular}

\section*{Empirics}
\subsection*{Operationalization}
To evaluate the theoretical expectations I focus on the international loan environment in Africa since the year 2000. Loan value, or the total sum of money being loaned to an African country, is the dependent variable. The main explanatory variables of interest are recipient need and the clarity of loan conditionality. Recipient need is captured by gross domestic product (GDP) per capita while the clarity of loan conditionality will be an indicator for how ``offical'' aid is. Thus, the empirical model is:

$\text{Loan Value}= \beta_0 + \beta_1 \text{GDP per capita}+ \beta_2 \text{Loan Conditionality}+ \gamma_i x_i + \varepsilon$,

where $x_i$ is a vector of controls.

\subsection*{Data \& Variables}
For data on international loans, I use data from William \& Mary's ``Aiddata'' lab for Chinese loans (\cite{custer2021}) and data from the OECD on Western loans (\cite{oecd2022}). GDP per capita stems from the World Bank (\cite{bank2022}) while I am currently exploring measures for loan conditionality. The most overly-simplistic (and not accurate) idea is to label all Chinese loans as ``low'' or ``vague'' in their conditionality while rating all Western loans as ``high'' or ``clear'' in their loan conditions. There is data on World Bank conditionality, though, (see \cite{hernandez2017}, for example) but I have yet to analyze the Chinese data for conditionality measures.

\subsection*{Estimation}
I plan to estimate the regression equation above using panel data and OLS with country-fixed effects and clustered standard errors by country.

\pagebreak
\section*{Appendix}
\subsection*{Figure 1}
\begin{tikzpicture}[scale=1,font=\footnotesize]
    \tikzstyle{solid node}=[circle,draw,inner sep=1.5,fill=black]
    \tikzstyle{hollow node}=[circle,draw,inner sep=1.5]
    \tikzstyle{level 1}=[level distance=2cm,sibling distance=2cm]
    \tikzstyle{level 2}=[level distance=2cm,sibling distance=2cm]
    \tikzstyle{level 3}=[level distance=2cm,sibling distance=2cm]
    \tikzstyle{level 3}=[level distance=2cm,sibling distance=2cm]
    \tikzstyle{level 3}=[level distance=2cm,sibling distance=2cm]
    \tikzstyle{level 3}=[level distance=2cm,sibling distance=2cm]
    \node(0)[solid node,label=above:{$D_1$},label=below:{$x_1$}]{}
        child{node{}}
        child{node[solid node,label=above:{$R$}]{}
            child{node[hollow node,label=below:{$1-x_1,\; x_1,\; 0$}]{} edge from parent node[left]{Accept} edge from parent node[right,xshift=10mm]{Reject} edge from parent node[right,xshift=15mm,yshift=30mm]{$0\leq x_{t}<1,\;t=1$}}
                child{node[solid node,label=above:{$D_2$},label=below:{$x_2$}]{}
                    child{node{}}
                        child{node[solid node,label=above:{$R$}]{}
                                child{node[hollow node,label=below:{$0,\; \delta \gamma x_2,\; 1-\delta \gamma x_2$}]{} edge from parent node[left]{Accept} edge from parent node[right,xshift=10mm]{Reject} edge from parent node[right,xshift=15mm,yshift=35mm]{$0\leq \delta<1,\;0\leq \gamma <1,\;t=2$}}
                                    child{node[hollow node,label=below:{$0,\;0,\;0$}]{}
        }
        }
        }
        };
    \draw[dashed](0-1)to(0-2);
    \path (-1,-6) node (p1) {};
    \path (5,-2) node (p2) {};
    \draw[dashed] (p1) to (p2);
    \path (1,-6) node (p3) {};
    \path (3,-6) node (p4) {};
    \draw[dashed] (p3) to (p4);
\end{tikzpicture}

The utility functions of the players are simply:
\begin{align*}
    U_R&:\;x_t\\
    U_{D_{t}}&:\;-x_t\;
\end{align*}
where $t\in {1,2}$ and $x_t$ is less than one but greater than or equal to zero. Making $x_t$ less than one allows for the donors to profit from any loan they make, enabling the offering of loans in the first place.

The parameters $\delta$ and $\gamma$ are all greater than or equal to zero and bounded by one. These modeling specifications seem to be without loss of generality, implying that the results are not dependent on these decisions.

Solving the game via backwards induction:

\begin{center}
\begin{tabular}{c l}
    $\delta \gamma x_2 >0$ & $R$ accepts $x_2$ iff this inequality is true in period two. \\
    $x_2= \epsilon$ & $D_2$'s offer, where $\epsilon$ is infinitely small but greater than 0.\\
    $x_1>\epsilon$ & The condition for $R$ to accept in period one.\\
    $x_1=2 \epsilon$ & $D_1$'s offer, which $R$ accepts.\\
\end{tabular}
\end{center}

Ultimately, $x_1>x_2$. Thus, competition alone increases the loan payoff for the recipient. These results are augmented by severity of need ($\delta$) and the clarity of conditionality ($\gamma$). Smaller values of either of these parameters coincides with smaller loan payoffs to the recipient in period two, which leads to smaller counteroffers by the first donor in period one.